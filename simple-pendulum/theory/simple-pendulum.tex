\documentclass[12pt]{article}

\title{Simple Pendulum Simulation}
\author{Karan Kinariwala - 19PHY77025}

\usepackage{amsmath}
\usepackage[margin=1.0in]{geometry}
\usepackage{graphicx}
\graphicspath{ {./images/} }
\usepackage{hyperref}
\usepackage{xcolor}
\usepackage{float}


\begin{document}
\maketitle

\begin{center}
    \textcolor{blue}{\underline{\href{https://github.com/kkin1995/simulations/}{Link to Code}}}
\end{center}



In this project, we use Python to simulate the Simple Pendulum, with and without the linear approximation.
In the simulation, we will solve the second order ordinary differential equation using a numerical integration
technique known as the Runge - Kutta 4th Order Integrator (RK4). \newline

\section{Theory}

First, we will obtain the equation of motion of the simple pendulum using Lagrangian Mechanics and the Principle of Least
Action. \newline

\begin{figure}
    \caption{Simple Pendulum}
    \includegraphics[width = \textwidth]{pendulum}
    \label{fig:sp}
\end{figure}

In the Cartesian Coordinate System, the Simple Pendulum system has two degrees of freedom whereas
in Spherical Polar Coordinate System, the Simple Pendulum system has only one degree of freedom as shown in Figure \ref{fig:sp} \newline

The equations to transform from Cartesian to Spherical Coordinates are given as,

\begin{equation} \label{eqn1}
    \begin{split}
        x &= l\sin{\theta} \\ 
        y &= l - l\cos{\theta} \\
    \end{split}
\end{equation}

Now taking the first time derivative of equation \ref{eqn1},

\begin{equation} \label{eqn2}
    \begin{split}
        \dot{x} &= l \dot{\theta} \cos{\theta}  \\
        \dot{y} &= l \dot{\theta} \sin{\theta}  \\
    \end{split}
\end{equation}

The Kinetic Energy of the Simple Pendulum system is given as,

\begin{equation} \label{eqn3}
    \begin{split}
        T &= \frac{1}{2} m v^{2} \\
        &= \frac{1}{2} m ( \dot{x}^{2} + \dot{y}^{2} ) \\
    \end{split}
\end{equation}

where, $m$ is the mass of the bob and $v$ is it's velocity.

\newpage

Substituting equation \ref{eqn2} in equation \ref{eqn3} we obtain,

\begin{equation} \label{eqn4}
    \begin{split}
        T &= \frac{1}{2} m ( l^{2} \dot{\theta}^{2} \cos^{2}{\theta} + l^{2} \dot{\theta}^{2} \sin^{2}{\theta} ) \\
        &= \frac{1}{2} m  l^{2} \dot{\theta}^{2} ( \cos^{2}{\theta} + \sin^{2}{\theta} ) \\
        T &= \frac{1}{2} m  l^{2} \dot{\theta}^{2} \\
    \end{split}
\end{equation}

The Potential Energy of the Simple Pendulum system is given as,

\begin{equation} \label{eqn5}
    \begin{split}
        V &= mgy \\
        &= mg ( l - l\cos{\theta} ) \\
        V &= mgl ( 1 - \cos{\theta} ) \\
    \end{split}
\end{equation}

where, $m$ is the mass of the bob and $g$ is the acceleration due to gravity.

Now, from Lagrangian Mechanics, we know that the Lagrangian is given by,

\begin{equation} \label{eqn6}
    \begin{split}
       L &= T - V \\
    \end{split}
\end{equation}

Substituting equations \ref{eqn4} and \ref{eqn5} into equation \ref{eqn6}, we obtain,

\begin{equation} \label{eqn7}
    \begin{split}
        L &= \frac{1}{2} m  l^{2} \dot{\theta}^{2} - mgl ( 1 - \cos{\theta} ) \\
    \end{split}
\end{equation}

With the assumption that there are no external forces, we can use the Principle of Least Action to
obain Lagrange's Differential Equation. Since there is only one degree of freedom, there will be only
one second order differential equation to solve.

\begin{equation} \label{eqn8}
    \begin{split}
        \frac{d}{dt} \left( \frac{\partial{L}}{\partial{\dot{\theta}}} \right) - \frac{\partial{L}}{\partial{\theta}} = 0                   
    \end{split}
\end{equation}

We shall first calculate the two terms in equation \ref{eqn8} and then substitute it back into equation \ref{eqn8}.

Calculating the first term,

\begin{equation} \label{eqn9}
    \begin{split}
        \frac{\partial{L}}{\partial{\dot{\theta}}} &= m l^{2} \dot{\theta}^{2} \\
        \frac{d}{dt} \left( \frac{\partial{L}}{\partial{\dot{\theta}}} \right) &= m l^{2} \ddot{\theta}^{2} \\
    \end{split}
\end{equation}

Now, calculating the second term,

\begin{equation} \label{eqn10}
    \begin{split}
        \frac{\partial{L}}{\partial{\theta}} &= -mgl\sin{\theta} \\
    \end{split}
\end{equation}

Now, substituting equation \ref{eqn9} and equation \ref{eqn10} into equation \ref{eqn8},

\begin{equation} \label{eqn11}
    \begin{split}
        & m l^{2} \ddot{\theta}^{2} + mgl\sin{\theta} = 0 \\
        & m l^{2} \ddot{\theta}^{2} = - mgl\sin{\theta} \\
        & l \ddot{\theta} = - g \sin{\theta} \\
        & \ddot{\theta} = - \frac{g}{l} \sin{\theta} \\
    \end{split}
\end{equation}

Equation \ref{eqn11} is the equation of motion for a Simple Pendulum. In the case that we have
to solve this equation exactly, we assume that the angle $\theta$ is small and take the linear tern
from the taylor expansion of $\sin{\theta}$.

The taylor expansion of $\sin{\theta}$ is given as,

\begin{equation} \label{eqn12}
    \begin{split}
       \sin{\theta} &= \theta - \frac{\theta^{3}}{6} + \frac{\theta^{5}}{120} + ... \\
       \sin{\theta} &= \theta \\ 
    \end{split}
\end{equation}

Substituting equation \ref{eqn12} in to equation \ref{eqn11} to obtain an approximate linear differential equation,

\begin{equation} \label{eqn13}
    \begin{split}
        \ddot{\theta} = - \frac{g}{l} \theta \\
    \end{split}
\end{equation}

The solution to equation \ref{eqn13} is well - known. It can be expressed as an exponential. 
We also assume the following initial condition. At time $t = 0$, the angle $\theta$ is 
$ \theta_{0} $ Assume the solution:

\begin{equation} \label{eqn14}
    \begin{split}
        \theta &= A e^{\lambda t} \\
        \dot{\theta} &= A \lambda e^{\lambda t} \\
        \ddot{\theta} &= A \lambda^{2} e^{\lambda t} \\
        \ddot{\theta} &= \lambda^{2} \theta \\
    \end{split}
\end{equation}

Now, substituting equation \ref{eqn14} back into equation \ref{eqn13},

\begin{equation} \label{eqn15}
    \begin{split}
        \lambda^{2} \theta &= - \frac{g}{l} \theta  \\
        \lambda^{2} &= - \frac{g}{l} \\
        \lambda &= \sqrt{- \frac{g}{l}} \\
        \lambda &=  i \omega \\
    \end{split}
\end{equation}

where $\omega = \sqrt{\frac{g}{l}}$ is the frequency of oscillation of the Simple Pendulum.
Now, we substitute equation \ref{eqn15} back into our assumed solution to obtain,

\begin{equation} \label{eqn16}
    \begin{split}
       \theta &= A e^{i \omega t} \\
    \end{split}
\end{equation}

As per our initial condition,

\begin{equation} \label{eqn17}
    \begin{split}
      \theta_{0} &= A e^{i \omega 0} \\
      \theta_{0} &= A \\
    \end{split}
\end{equation}

Therefore, our solution becomes,

\begin{equation} \label{eqn18}
    \begin{split}
        \theta = \theta_{0} e^{i \omega t}
    \end{split}
\end{equation}

Taking the real part of the exponential, we obtain,

\begin{equation} \label{eqn19}
    \begin{split}
        \theta = \theta_{0} \cos{ (\omega t) }
    \end{split}
\end{equation}

The Non-Linear Differential Equation (Equation \ref{eqn11}) will be numerically solved using the
Runge - Kutta 4th Order (RK-4) Integrator in Python.

Additionally, in the simulation, we will also show the difference between the solutions for
the Linear and Non-Linear Differential Equations for various angles. \newline

The code for the implementation of the Runge - Kutta 4th Order Integrator is given in the file "simple\_pendulum.py"
in the repository which is linked above. The code can also be found at this \textcolor{blue}{\underline{\href{https://github.com/kkin1995/simulations/simple_pendulum.py}{Link}}}. \newline

The fie "simple\_pendulum.py" contains a python class named "SimplePendulum" which has methods for solving and plotting the equation of motion using
the RK-4 Integrator and plotting the linear approximate solution \newline

In addition to the implementation of the integrator, there are functional implentations for plotting the phase space trajectory and the energy
of the simple pendulum. \newline

\section{Comparing the Numerical Solution to the Linear Approximation}

Let us consider a Simple Pendulum oscillating on the planet Earth ($g = 9.81$) with an initial angle of $10^\circ$. Simulating and plotting
$\theta$ vs time on the same plot we get,

\begin{figure}[H]
    \centering
    \caption{Initial Angle = $10^\circ$}
    \includegraphics[width = \textwidth]{time_theta_10}
    \label{fig:theta10}
\end{figure}

As we see from the plot, the two graphs are more or less coninciding which shows us that the small angle approximation is a good approximation for
very small angles.

Now, if we use initial angle of $45^\circ$, we get the following,

\begin{figure}[H]
    \centering
    \caption{Initial Angle = $45^\circ$}
    \includegraphics[width = \textwidth]{time_theta_45}
    \label{fig:theta45}
\end{figure}

Here we see that the solution to our differential equation \ref{eqn11} is different to the solution of the linear approximated differential equation.

\section{Phase Space Trajectory}

Phase Space is the space where all possible states of a system is represented. A single point is phase space corresponds to a unique state of the system.

In this section, we investigate how the trajectory of the Simple Pendulum will vary in phase space when it starts at different initial angles.
Let us consider four different initial angles: $10^\circ$, $25^\circ$, $45^\circ$ and $90^\circ$. We shall use the file "phase\_space.py" found at this
\textcolor{blue}{\underline{\href{https://github.com/kkin1995/simulations/phase_space.py}{Link}}}.

\begin{figure}[H]
    \centering
    \caption{Phase Space Trajectory}
    \includegraphics[width = \textwidth]{phase_space}
    \label{fig:phasespace}
\end{figure}

The phase space trajectory (as in Lagrangian Mechanics) is the plot of the velocity of the system as a function of position. In this case, the velocity is
$\dot{\theta}$ and the position is $\theta$.

Along the phase space trajectories shownn above, the energy of the system is conserved. In other words, they are contour lines of equal energy.

Another defining feature of the phase space trajectory is that the trajectories are closed. This means that the motion of the system is periodic.

However, if the pendulum begins to oscillate at an initial angle close to $180^\circ$, we will get a discontinuous phase space trajectory as shown below:

\begin{figure}[H]
    \centering
    \caption{Phase Space Trajectory - $179.9^\circ$}
    \includegraphics[width = \textwidth]{phase_space_dis}
    \label{fig:phasespacediscontinuous}
\end{figure}

\section{Conservation of Energy}
Earlier, in the section on phase space trajectories, we had said that the trajectories are contour lines of equal energy which means that the energy of the system is
conserved throughout it's motion. In this section, we will plot the Kinetic Energy, Potential Energy and Total Energy to show that the Kinetic and Potential Energy oscillate
in just the right way to conserve the Total Energy.

We will use the function "plot\_energy()" which is defined in the file "energy.py" and can be found at this
\textcolor{blue}{\underline{\href{https://github.com/kkin1995/simulations/energy.py}{Link}}}.

We will plot the Kinetic, Potential and Total Energy for the pendulum with an inital angle of $10^\circ$.

\begin{figure}[H]
    \centering
    \caption{Energy Plot}
    \includegraphics[width = \textwidth]{energy}
    \label{fig:energy}
\end{figure}

As we can see from the plot, the green line is the Total Energy and it is constant for the entire motion of the pendulum, which is
indicative of the conservation of energy.

\end{document}