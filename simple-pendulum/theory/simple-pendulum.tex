\documentclass[12pt]{article}

\title{Simple Pendulum Simulation}
\author{Karan}

\usepackage{amsmath}
\usepackage[margin=1.0in]{geometry}



\begin{document}
\maketitle

In this project, we use Python to simulate the Simple Pendulum, with and without the linear approximation.
In the simulation, we will solve the second order ordinary differential equation using a numerical integration
technique known as the Runge - Kutta 4th Order Integrator (RK4). \newline

First, we will obtain the equation of motion of the simple pendulum using Lagrangian Mechanics and the Principle of Least
Action. \newline

[PLACE FIGURE HERE]

In the Cartesian Coordinate System, the Simple Pendulum system has two degrees of freedom whereas
in Spherical Polar Coordinate Systems, the Simple Pendulum system has only one degree of freedom as shown in the figure. \newline

The equations to transform from Cartesian to Spherical Coordinates are given as,

\begin{equation} \label{eqn1}
    \begin{split}
        x &= l\sin{\theta} \\
        y &= l - l\cos{\theta} \\
    \end{split}
\end{equation}

Now taking the first time derivative of equation \ref{eqn1},

\begin{equation} \label{eqn2}
    \begin{split}
        \dot{x} &= l \dot{\theta} \cos{\theta}  \\
        \dot{y} &= l \dot{\theta} \sin{\theta}  \\
    \end{split}
\end{equation}

The Kinetic Energy of the Simple Pendulum system is given as,

\begin{equation} \label{eqn3}
    \begin{split}
        T &= \frac{1}{2} m v^{2} \\
        &= \frac{1}{2} m ( \dot{x}^{2} + \dot{y}^{2} ) \\
    \end{split}
\end{equation}

where, $m$ is the mass of the bob and $v$ is it's velocity.

\newpage

Substituting equation \ref{eqn2} in equation \ref{eqn3} we obtain,

\begin{equation} \label{eqn4}
    \begin{split}
        T &= \frac{1}{2} m ( l^{2} \dot{\theta}^{2} \cos^{2}{\theta} + l^{2} \dot{\theta}^{2} \sin^{2}{\theta} ) \\
        &= \frac{1}{2} m  l^{2} \dot{\theta}^{2} ( \cos^{2}{\theta} + \sin^{2}{\theta} ) \\
        T &= \frac{1}{2} m  l^{2} \dot{\theta}^{2} \\
    \end{split}
\end{equation}

The Potential Energy of the Simple Pendulum system is given as,

\begin{equation} \label{eqn5}
    \begin{split}
        V &= mgy \\
        &= mg ( l - l\cos{\theta} ) \\
        V &= mgl ( 1 - \cos{\theta} ) \\
    \end{split}
\end{equation}

where, $m$ is the mass of the bob and $g$ is the acceleration due to gravity.

Now, from Lagrangian Mechanics, we know that the Lagrangian is given by,

\begin{equation} \label{eqn6}
    \begin{split}
       L &= T - V \\
    \end{split}
\end{equation}

Substituting equations \ref{eqn4} and \ref{eqn5} into equation \ref{eqn6}, we obtain,

\begin{equation} \label{eqn7}
    \begin{split}
        L &= \frac{1}{2} m  l^{2} \dot{\theta}^{2} - mgl ( 1 - \cos{\theta} ) \\
    \end{split}
\end{equation}

With the assumption that there are no external forces, we can use the Principle of Least Action to
obain Lagrange's Differential Equation. Since there is only one degree of freedom, there will be only
one second order differential equation to solve.

\begin{equation} \label{eqn8}
    \begin{split}
        \frac{d}{dt} \left( \frac{\partial{L}}{\partial{\dot{\theta}}} \right) - \frac{\partial{L}}{\partial{\theta}} = 0                   
    \end{split}
\end{equation}

We shall first calculate the two terms in equation \ref{eqn8} and then substitute it back into equation \ref{eqn8}.

\newpage

Calculating the first term,

\begin{equation} \label{eqn9}
    \begin{split}
        \frac{\partial{L}}{\partial{\dot{\theta}}} &= m l^{2} \dot{\theta}^{2} \\
        \frac{d}{dt} \left( \frac{\partial{L}}{\partial{\dot{\theta}}} \right) &= m l^{2} \ddot{\theta}^{2} \\
    \end{split}
\end{equation}

Now, calculating the second term,

\begin{equation} \label{eqn10}
    \begin{split}
        \frac{\partial{L}}{\partial{\theta}} &= -mgl\sin{\theta} \\
    \end{split}
\end{equation}

Now, substituting equation \ref{eqn9} and equation \ref{eqn10} into equation \ref{eqn8},

\begin{equation} \label{eqn11}
    \begin{split}
        & m l^{2} \ddot{\theta}^{2} + mgl\sin{\theta} = 0 \\
        & m l^{2} \ddot{\theta}^{2} = - mgl\sin{\theta} \\
        & l \ddot{\theta} = - g \sin{\theta} \\
        & \ddot{\theta} = - \frac{g}{l} \sin{\theta} \\
    \end{split}
\end{equation}

Equation \ref{eqn11} is the equation of motion for a Simple Pendulum. In the case that we have
to solve this equation exactly, we assume that the angle $\theta$ is small and take the linear tern
from the taylor expansion of $\sin{\theta}$.

The taylor expansion of $\sin{\theta}$ is given as,

\begin{equation} \label{eqn12}
    \begin{split}
       \sin{\theta} &= \theta - \frac{\theta^{3}}{6} + \frac{\theta^{5}}{120} + ... \\
       \sin{\theta} &= \theta \\ 
    \end{split}
\end{equation}

Substituting equation \ref{eqn12} in to equation \ref{eqn11} to obtain an approximate linear differential equation,

\begin{equation} \label{eqn13}
    \begin{split}
        \ddot{\theta} = - \frac{g}{l} \theta \\
    \end{split}
\end{equation}

The solution to equation \ref{eqn13} is well - known. It can be expressed as an exponential. 
We also assume the following initial condition. At time $t = 0$, the angle $\theta$ is 
$ \theta_{0} $ Assume the solution:

\begin{equation} \label{eqn14}
    \begin{split}
        \theta &= A e^{\lambda t} \\
        \dot{\theta} &= A \lambda e^{\lambda t} \\
        \ddot{\theta} &= A \lambda^{2} e^{\lambda t} \\
        \ddot{\theta} &= \lambda^{2} \theta \\
    \end{split}
\end{equation}

\newpage

Now, substituting equation \ref{eqn14} back into equation \ref{eqn13},

\begin{equation} \label{eqn15}
    \begin{split}
        \lambda^{2} \theta &= - \frac{g}{l} \theta  \\
        \lambda^{2} &= - \frac{g}{l} \\
        \lambda &= \sqrt{- \frac{g}{l}} \\
        \lambda &=  i \omega \\
    \end{split}
\end{equation}

where $\omega = \sqrt{\frac{g}{l}}$ is the frequency of oscillation of the Simple Pendulum.
Now, we substitute equation \ref{eqn15} back into our assumed solution to obtain,

\begin{equation} \label{eqn16}
    \begin{split}
       \theta &= A e^{i \omega t} \\
    \end{split}
\end{equation}

As per our initial condition,

\begin{equation} \label{eqn17}
    \begin{split}
      \theta_{0} &= A e^{i \omega 0} \\
      \theta_{0} &= A \\
    \end{split}
\end{equation}

Therefore, our solution becomes,

\begin{equation} \label{eqn18}
    \begin{split}
        \theta = \theta_{0} e^{i \omega t}
    \end{split}
\end{equation}

Taking the real part of the exponential, we obtain,

\begin{equation} \label{eqn19}
    \begin{split}
        \theta = \theta_{0} \cos{ (\omega t) }
    \end{split}
\end{equation}

The Non-Linear Differential Equation (Equation \ref{eqn11}) will be numerically solved using the
Runge - Kutta 4th Order (RK-4) Integrator in Python.

Additionally, in the simulation, we will also show the difference between the solutions for
the Linear and Non-Linear Differential Equations for various angles.

\end{document}